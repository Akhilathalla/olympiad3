\documentclass{article}
\usepackage{gvv-book}
\usepackage{gvv}
\begin{document}
\begin{enumerate}
\item Given triangle $ABC$ the point $J$ is the centre of the excircle opposite the vertex $A$. This excircle is tangent to the side $BC$ at $M$, and to the lines $AB$ and $AC$ at $K$ and $L$, respectively. The lines $LM$ and $BJ$ meet at $F$, and the lines $KM$ and $CJ$ meet at $G$. Let $S$ be the point of intersection of the lines $AF$ and $BC$, and let $T$ be the point of intersection of the lines $AG$ and $BC$. Prove that $M$ is the midpoint of $ST$.
(The $excircle$ of $ABC$ opposite the vertex $A$ is the circle that is tangent to the line segment $BC$, to the ray $AB$ beyond $B$, and to the ray $AC$ beyond $C$.)
\item  Let $n\geq{3}$ be an integer, and let $a_{2}$, $a_{3}$,\dots, $a_{n}$  be positive real numbers such that $a_{2}a_{3}$ \dots $a_{n}$ = $1$. Prove that
	\begin{align*}
	(1+a_{2}) ^{2}  (1+a_{3}) ^{3} \dots (1+a_{n}) ^{n} > n^{n}
	\end{align*}
\item The liar's guessing game is a game played between two players $A$ and $B$. The rules of the game depend on two positive integers $k$ and $n$ which are known to both players. At the start of the game $A$ chooses integers $x$ and $N$ with $1\leq{x}\leq{N}$. Player $A$ keeps $x$ secret, and truthfully tells $N$ to player $B$. Player $B$ now tries to obtain information about $x$ by asking player $A$ questions as follows: each question consists of $B$ specifying an arbitrary set $S$ of positive integers (possibly one specified in some previous question), and asking $A$ whether $x$ belongs to $S$. Player $B$ may ask as many such questions as he wishes. After each question, player $A$ must immediately answer it with yes or no, but is allowed to lie as many times as she wants; the only restriction is that, among any $k + 1$ consecutive answers, at least one answer must be truthful.
	After $B$ has asked as many questions as he wants, he must specify a set $X$ of at most $n$ positive integers. If $x$ belongs to $X$, then $B$ wins; otherwise, he loses. Prove that: \\
1. If $n\geq{2^{k}}$, then $B$ can guarantee a win.

		2. For all sufficiently large $k$, there exists an integer $n\geq{1.99^{k}}$ such that $B$ cannot guarantee a win.
	\item Find all functions $f$:$\mathbb{Z}$ $\rightarrow$ $\mathbb{Z}$  such that, for all integers $a$, $b$, $c$ that satisfy $a+b+c = 0$, the following equality holds:
		\begin{align*}
		f \brak a^2+f \brak b^2+f \brak c^2=2f \brak af \brak b+2f \brak b f \brak c+2f \brak c f \brak a.
		\end{align*}
		(Here $\mathbb{Z}$ denotes the set of integers.)
	\item  Let $ABC$ be a triangle with $\angle{BCA}=90\degree$, and let $D$ be the foot of the altitude from $C$. Let $X$ be a point in the interior of the segment $CD$. Let $K$ be the point on the segment $AX$ such that $BK = BC$. Similarly, let $L$ be the point on the segment $BX$ such that $AL = AC$. Let $M$ be the point of intersection of $AL$ and $BK$.
		Show that $MK = ML$.
	\item  Find all positive integers $n$ for which there exist non-negative integers $a_{1}$, $a_{2}$, \dots, $a_{n}$ such that
		\begin{align*}
		\frac{1}{2^{a_{1}}}+\frac{1}{2^{a_{2}}}+ \dots +\frac{1}{2^{a_{n}}}=\frac{1}{3^{a_{1}}}+\frac{1}{3^{a_{2}}}+ \dots +\frac{n}{3^{a_{n}}}=1.
		\end{align*}
\end{enumerate}
\end{document}
